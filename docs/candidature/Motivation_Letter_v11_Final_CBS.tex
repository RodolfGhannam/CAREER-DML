\documentclass[11pt,a4paper]{article}

% --- Packages ---
\usepackage[utf8]{inputenc}
\usepackage[T1]{fontenc}
\usepackage{lmodern}
\usepackage[margin=2.5cm]{geometry}
\usepackage{setspace}
\usepackage{parskip}
\usepackage[hidelinks]{hyperref}
\usepackage{fancyhdr}

% --- Page Style ---
\pagestyle{fancy}
\fancyhf{}
\renewcommand{\headrulewidth}{0pt}
\fancyfoot[C]{\thepage}

% --- Spacing ---
\setstretch{1.15}
\setlength{\parskip}{0.5em}

\begin{document}

% --- Header ---
\begin{flushright}
    \textbf{Rodolf Mikel Ghannam Neto}\\
    \href{mailto:rodolf@cical.com.br}{rodolf@cical.com.br}
\end{flushright}

\vspace{1.5em}

To the PhD Admissions Committee,

Department of Strategy and Innovation

Copenhagen Business School

\vspace{1.5em}

\textbf{Subject: Motivation for the PhD Position in Strategy and Innovation}

\vspace{1.5em}

Dear Committee Members,

There is a gap between what practitioners observe in the field and what the academic literature is equipped to measure. I saw it firsthand---employees with nearly identical backgrounds diverging sharply in their development trajectories, in ways that no standard analysis could explain. Rather than accept that gap, I decided to try to close it. Without a lab, without a supervisor, and without institutional affiliation, I built a causal inference framework from scratch, validated it against ground truth, and published the code. That framework now needs what only a PhD program can provide: population-scale data, rigorous supervision, and the time to do the work properly. I believe CBS is the ideal environment for this next intellectual chapter, and I would be grateful for the opportunity.

For over fifteen years, I have led companies through technological disruption---from founding a food distribution company that supplied Brazil's largest retail chains, to directing corporate strategy for a diversified conglomerate spanning automotive, real estate, and financial services. But my career has been defined not merely by managing change, but by an evolving drive to \textit{understand} it. That drive eventually led me to the question at the heart of my research: what is the causal impact of AI adoption on individual career trajectories? With 34.7\% of the Danish workforce already exposed to generative AI in tasks representing 20\% or more of their work hours (OECD, 2024), this question has become a first-order policy concern for Denmark and the broader European labor market.

To address it, I developed and validated \textsc{career-dml}, a framework for causal inference on sequential career data. The enclosed research proposal and working paper detail three core contributions. First, the \textbf{Sequential Embedding Ordering Phenomenon}---a robust empirical finding, observed across three distinct scenarios, that challenges the direct application of some causal representation learning methods to socio-economic data. Second, the \textbf{Signal-to-Noise Frontier}---a characterization of the sample size requirements for detecting realistic causal effects, providing a rigorous justification for the necessity of large-scale administrative data. Third, a \textbf{bridge between the two cultures of modern econometrics}---causal Machine Learning and structural modeling---showing that the learned embeddings serve as non-parametric analogs of the latent variables that structural economists postulate but rarely observe directly, from human capital accumulation (\`a la Ben-Porath, 1967) to latent occupational types (\`a la Keane \& Wolpin, 1997). Linear probing experiments confirm that these representations encode interpretable economic structure without being explicitly trained on it. The framework strategically uses pre-2022 occupational AI exposure metrics (Felten et al., 2021 AIOE scores), providing a clean baseline to later disentangle the effects of the Generative AI shock within the Danish registers.

The results are concrete. In a semi-synthetic laboratory calibrated with real US labor market data (NLSY79 transition matrices and Felten AIOE scores), the framework recovers the true causal effect with only 7.6\% bias---an order-of-magnitude improvement over all seven benchmarked methods, from classical Heckman (945\%) to modern LASSO and Random Forest approaches. A formal GATES heterogeneity test confirms statistically significant skill-biased treatment effects. But this is precisely where the work reaches its productive limit. The next essential step---crossing the Signal-to-Noise Frontier with population-scale data---requires the Danish IDA registers: N $>$ 1M individuals, T $>$ 30 years, no sampling bias. With this data, the framework could produce the first population-level causal estimates of how AI adoption reshapes individual career trajectories across Denmark---identifying which workers benefit, which are displaced, and why. Each limitation of my current work is not a weakness, but a direct motivation for the next step---a step that, to my knowledge, can only be taken at CBS.

The origin of this research is deeply personal. The system I built at Grupo CICAL contributed to earning General Motors' highest dealership distinction for four consecutive years---but more importantly, it showed me in the field exactly what the literature theorizes: that human capital forms differently across individuals, and that understanding this heterogeneity is the key to effective intervention.

The opportunity to pursue this work within the Department of Strategy and Innovation, under the guidance of Professor Tom Gard, offers a natural alignment with my research trajectory. The department's Digital Transformations agenda and its privileged access to the IDA registers create a research environment that, to my knowledge, exists at no other institution. I look forward to contributing to this environment and learning from the faculty's collective expertise in labor markets, innovation, and applied econometrics. As Ayrton Senna once said, ``Whoever you are, whatever your position in life, always aim for great strength, great determination, and always do everything with great love and great faith---one day you will get there.'' It is with this spirit that I approach this next chapter.

I am aware that my path is unconventional. I do not come from a traditional academic background, and I have much to learn. But it took a certain courage to look at the gap between what practitioners observe and what the literature measures, and to decide---without institutional support or academic affiliation---to build a framework to close it. I come with real questions born from real experience, a creative instinct for connecting ideas across disciplines, and the willingness to do whatever the work demands. I would be honored to bring this commitment to CBS. Thank you for your time and consideration.

\vspace{2em}

Sincerely,

\vspace{1.5em}

Rodolf Mikel Ghannam Neto

\end{document}
