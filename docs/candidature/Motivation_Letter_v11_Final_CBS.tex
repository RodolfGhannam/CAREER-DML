\documentclass[11pt,a4paper]{article}

% --- Packages ---
\usepackage[utf8]{inputenc}
\usepackage[T1]{fontenc}
\usepackage{lmodern}
\usepackage[margin=2.5cm]{geometry}
\usepackage{setspace}
\usepackage{parskip}
\usepackage[hidelinks]{hyperref}
\usepackage{fancyhdr}

% --- Page Style ---
\pagestyle{fancy}
\fancyhf{}
\renewcommand{\headrulewidth}{0pt}
\fancyfoot[C]{\thepage}

% --- Spacing ---
\setstretch{1.15}
\setlength{\parskip}{0.5em}

\begin{document}

% --- Header ---
\begin{flushright}
    \textbf{Rodolf Mikel Ghannam Neto}\\
    \href{mailto:rodolf@cical.com.br}{rodolf@cical.com.br}
\end{flushright}

\vspace{1.5em}

To the PhD Admissions Committee,

Department of Strategy and Innovation

Copenhagen Business School

\vspace{1.5em}

\textbf{Subject: Motivation for the PhD Position in Strategy and Innovation}

\vspace{1.5em}

Dear Committee Members,

I am writing to express my profound interest in the PhD program at the Department of Strategy and Innovation. My career, spanning over 15 years in executive roles, has been defined by a relentless drive to leverage data and technology to understand and shape organizational dynamics. This journey has led me from implementing proprietary data-driven methodologies in customer service---which revealed the deep-seated heterogeneity in employee development---to the independent research that forms the basis of my application. My work has now reached a critical juncture where the practitioner's questions demand the full rigor of academic inquiry, and I believe CBS is the ideal environment for this next intellectual chapter.

My enclosed research proposal and working paper, \textit{CAREER-DML}, detail a validated framework for causal inference on career trajectories. This work makes three core contributions that I intend to build upon during my PhD. First, it documents the \textbf{Sequential Embedding Ordering Phenomenon}, a novel empirical finding that challenges the direct application of some causal representation learning methods to socio-economic data. Second, it characterizes the \textbf{Signal-to-Noise Frontier}, providing a rigorous justification for the use of large-scale administrative data to detect realistic causal effects. Third, and most importantly, it builds a \textbf{bridge between the two cultures of modern econometrics}---causal Machine Learning and structural modeling---by showing how learned embeddings can serve as non-parametric analogs of classical latent variables like human capital. This work also strategically uses pre-2022 AI exposure metrics, providing a clean baseline to later disentangle the effects of the recent Generative AI shock, a topic of immense relevance to the Danish labor market.

My ambition is to be a scholar who is, in the words of the late Ayrton Senna, ``intellectually ambitious, methodologically sound, and technically consistent.'' The opportunity to pursue this research at CBS, under the potential supervision of Professor H.C. Kongsted, is unparalleled. His work on human capital and labor market dynamics provides the ideal intellectual anchor for my project. The department's focus on Digital Transformations and its access to world-class administrative data create a research environment that is perfectly aligned with my objectives.

I am confident that my unique background as a seasoned practitioner with a demonstrated capacity for independent, high-level quantitative research will allow me to make a significant contribution to the department and the broader academic community. Thank you for your time and consideration.

\vspace{2em}

Sincerely,

\vspace{1.5em}

Rodolf Mikel Ghannam Neto

\end{document}
